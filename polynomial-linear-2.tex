
\documentclass[11pt]{article}

\usepackage{url}
\usepackage{hyperref}
\usepackage{graphicx}
\usepackage{grffile}
\usepackage{xcolor}

\graphicspath{{figs/}}
\usepackage{enumitem}
\begin{document}


\title{Mathematics with the HP48G Calc}
\author{GNU Author}
\date{2006 February}
\maketitle


\section{Systems of Equations}
\subsection{System with one unknown}
\begin{enumerate}
\item Example (1)
\begin{enumerate}
\item $ 4x^2 +3x -3 =  0 $
\end{itemize}
\end{itemize}



\subsection{Polynomial system}
\begin{enumerate}
\item Example (1)
\begin{enumerate}
\item $ x^2  +4y^2 =  3 $
\item $ x   = -5 y^2  $
\item $ -z^2 = 3x $
\end{itemize}
\end{itemize}




\subsection{Linear system}
\begin{enumerate}
\item Example (1)
\begin{enumerate}
\item $ 2x   -4y  +5z =   101 $
\item $ 4x   +6y  -10z = -21 $
\item $ 13x  -2z = -30y $
\end{itemize}
\end{itemize}






\subsection{System with two equations}
\begin{enumerate}
\item Example (1)
\begin{enumerate}
\item $ 4  =  \sqrt{ 5 y x   }  $
\item $ 2 = \sqrt{ \frac{ 3 \dot x } {y}    } $
\end{itemize}
\end{itemize}





\subsection{System with more advanced equations}
\begin{enumerate}
\item Example (1)
\begin{enumerate}
\item $ z^3 = y^2 \cdot x   $
\item $ \sqrt{ x  } = log( 3 z^2 ) $
\item $  5 z x = y^x $
\end{itemize}
\end{itemize}




\end{document}




